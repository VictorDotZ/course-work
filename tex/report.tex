\documentclass[diploma]{nanolab2015}

\usepackage{float}
\usepackage{subfigure}
\usepackage{booktabs}
\usepackage{dcolumn}
\usepackage[flushleft]{threeparttable}
\usepackage{makecell}

\usepackage{enumitem}
\usepackage{xparse}
\newcounter{descriptcount}
\NewDocumentEnvironment{enumdescript}{O{}}{%
    \setcounter{descriptcount}{0}%
    \renewcommand*\descriptionlabel[1]{%
      \stepcounter{descriptcount}%
      \normalfont\bfseries ##1%
    }%
    \description%
  }%
  {\enddescription}



\DeclareMathOperator{\Attention}{Attention}
\DeclareMathOperator{\softmax}{softmax}
\DeclareMathOperator{\ReLU}{ReLU}
\DeclareMathOperator{\MSE}{MSE}
\DeclareMathOperator{\AdamW}{AdamW}


\begin{document}
\begin{titlepage}
    \begin{center}
        \large
        Федеральное государственное бюджетное образовательное учреждение
        высшего образования «Московский государственный университет имени
        М.В.Ломоносова»

        МЕХАНИКО-МАТЕМАТИЧЕСКИЙ ФАКУЛЬТЕТ

        \textbf{Кафедра Математической теории интеллектуальных систем}\\
        \vspace{4cm}
        \textsc{\Large Курсовая работа}\\[5mm]
        {\LARGE TODO}
    \end{center}
    \vspace{3cm}
    \null

    \begin{flushright}
        \normalsize \underline{Выполнил:}
        \\студент 431 группы
        \\Зенин В. О.
        \\ \underline{\hspace{4cm}}
    \end{flushright}
    \vspace{1cm}

    \begin{flushright}
        \normalsize \underline{Научный руководитель:}
        \\к.ф.-м.н., н.с Половников В. С.
        \\ \underline{\hspace{4cm}}
    \end{flushright}

    \vfill
    \begin{center}
        \textbf{Москва - 2023}
    \end{center}
\end{titlepage}
\setcounter{page}{2}
\tableofcontents{}  % оглавление
\newpage

\section{Введение}
Различные нейросетевые подходы и архитектуры могут использоваться для работы с изменяющимися во времени данными. В процессе своего обучения они способны извлекать сложные нелинейные зависимости из данных и генерировать своё предсказание, основываясь на этом.

Существует много временных рядов, связанных с финансами, например, цены различного рода активов. Также можно найти описание паттернов движения цены, полученные путём анализирования исторических данных биржевых котировок. Многие игроки используют их как основание для своих стратегий. Правила, образующиеся в результате найденных закономерностей, достаточно примитивны, как и сами паттерны. Если предположить, что кем-то найдена выгодная стратегия, то подобную способны найти и многие другие игроки, сводя на нет любую потенциальную выгоду. Вызывает интерес: способны ли нейронные сети находить паттерны и, тем самым, определять приносящие доход стратегии торговли, скрытые от большинства игроков.

Информация о классических финансовых инструментах во многом скрыта от игроков и хранится на биржах. Финансовые транзакции также скрыты за межбанковским обменом и не поддаются анализу. Однако существуют набирающую популярность криптофинансовые активы, информация о которых, по своей природе, намного более открыта и может быть использована для анализа движения цены.

Цель данной работы -- Исследование доступной публично информации о криптовалютах, построение нескольких архитектур нейронных сетей для анализа исторических данных и построение прогноза изменения будущей цены актива.


\paragraph*{Основными задачами курсовой являются:}
\begin{itemize}
    \item Изучение существующих данных в блокчейне Bitcoin.
    \item Практическая реализация моделей на базе рекуррентных нейронных сетей и архитектуре трансформера.
    \item Постановка задач предсказания движения цены как задачи регрессии и классификации.
    \item Сравнение полученных результатов между собой и определение перспектив подобных исследований.
\end{itemize}

\newpage
\section{Данные и извлечение признаков}
\subsection{Информация из сети}
В данной работе использованы дневные наблюдения о состоянии блокчейн сети Bitcoin с 10 мая 2020 года по 8 мая 2023 года, полученные с \href{https://www.blockchain.com/explorer/charts}{blockchain.com}. Некоторые базовые признаки также вычислены заранее поставщиком данных. Их описания собраны в таблице \ref{table:features}.\footnote{Признаки, отмеченные (*) имеют не более 3 пропущенных значений, которые восстановлены линейной интерполяцией.}

\subsection{Подготовка данных}
При работе с ценой актива часто используется логарифм цены,
$$
    \ln(\frac{x_{t}}{x_{t-1}})
$$
позволяющий перейти от абсолютных значений к относительным. Смысл данного преобразования заключается в том, что успешная стратегия приносит доход в результате изменения цен, умноженных на вложенный капитал и именно доход имеет ключевое значение.

Входные данные для нейронных сетей следует скалировать. Однако некоторые признаки в наших данных имеют количественную природу, что выражается в почти линейном росте. Например, абсолютное значение добытых на момент времени $t$ монет BTC. Больший смысл имеет изменение в добыче, так как оно потенциально способно дать сигнал о будущих движениях цены. Поэтому в нашем случае подобное преобразование уместно применить ко всем признакам.

\subsection{Формирование обучающего множества}
До логарифмирования имелось 1094 векторов, размерности 27 каждый. В результате преобразование наблюдение за первый день вырождается и остается 1093 вектора значений.

Для обучения использовались значения до 15 июня 2022 года. Для валидации -- с 15 июня 2022 года по 20 января 2023 года. Для теста -- с 20 января 2023 года по 8 мая 2023 года. Данные временные диапазоны выбраны чтобы обеспечить соотношение 70:20:10.

Целевой признак -- market-price.

Сформируем из данных следующие пары:
$$
    (X_{[m;t]}, Y_{t}),
$$
где $X_{[m;t]} = (x_{t-m}, x_{t-m-1}, ... , x_{t-2}, x_{t-1})$ -- подпоследовательность длины $m$, $x_t$ -- вектор признаков для момента времени $t$, $Y_t$ -- значение целевого признака.
Для задачи регрессии $Y_t = y_t$, где $y_t$ логарифм цены. Для задачи классификации
$
    Y_t =
    \begin{cases}
        1, y_t > 0 \\
        0, y_t \le 0
    \end{cases}
$

\renewcommand\theadalign{ll}

\begin{table}[h]
    \centering
    \caption{Признаки из блокчейн сети Bitcoin}
    \label{table:features}
    \begin{threeparttable}
        \begin{tabular}{|l|l|}
            \hline
            \thead{\bf Признак}                  & \thead{\bf Описание}                                        \\
            \hline
            total-bitcoins (*)                   & \makecell[l]{Количество добытых монет}                      \\
            market-price                         & \makecell[l]{Средняя цена в USD на крупнейших обменниках}   \\
            trade-volume                         & \makecell[l]{Объем обменянных BTC (USD)}                    \\
            \hline
            blocks-size                          & \makecell[l]{Размер сети блокчейна (MB)}                    \\
            avg-block-size                       & \makecell[l]{Средний размер блока (MB)}                     \\
            n-transactions-total                 & \makecell[l]{Количество транзакций}                         \\
            n-transactions-per-block             & \makecell[l]{Среднее число транзакций на блок}              \\
            n-payments-per-block                 & \makecell[l]{Среднее число наград за валидированный блок}   \\
            median-confirmation-time             & \makecell[l]{Медианное время, за которое обработанная       \\ транзакция добавляется к сети}                 \\
            avg-confirmation-time                & \makecell[l]{Среднее время, за которое обработанная         \\ транзакция добавляется к сети}                   \\
            \hline
            hash-rate                            & \makecell[l]{Мощность сети}                                 \\
            difficulty                           & \makecell[l]{Относительная мера сложности сети -- насколько \\ трудно валидировать очередной блок} \\
            transaction-fees                     & \makecell[l]{Выплаченные BTC за валидацию блоков}           \\
            transaction-fees-usd                 & \makecell[l]{Выплаченные USD за валидацию блоков}           \\
            fees-usd-per-transaction             & \makecell[l]{Среднея выплата в USD за                       \\ валидированную транзакцию} \\
            cost-per-transaction                 & \makecell[l]{Общий доход майнеров,                          \\ разделённый на количество транзакций}         \\
            \hline
            n-unique-addresses (*)               & \makecell[l]{Количество уникальных адресов,                 \\ используемых в сети}  \\
            n-transactions                       & \makecell[l]{Количество подтвержённых транзакций за день}   \\
            n-payments                           & \makecell[l]{Количество подтвержённых выплат за день}       \\
            % output-volume                    & \makecell[l]{}                                                                                       \\
            mempool-count                        & \makecell[l]{Количество неподтверждённых транзакций}        \\
            mempool-growth                       & \makecell[l]{Рост хранилищая неподтверждённых транзакций}   \\
            mempool-size                         & \makecell[l]{Размер хранилища неподтверждённых транзакций}  \\
            % utxo-count (*)                       & \makecell[l]{}                                                                                       \\
            n-transactions-excluding-popular     & \makecell[l]{Количество транзакций,                         \\ за исключением 100 самых популярных адресов} \\
            estimated-transaction-volume (*)     & \makecell[l]{Оценочная стоимость транзакций (BTC)}          \\
            estimated-transaction-volume-usd (*) & \makecell[l]{Оценочная стоимость транзакций (USD)}          \\
            \hline
        \end{tabular}
    \end{threeparttable}
\end{table}



\end{document}
